\documentclass{article}
%\usepackage{natbib}
\usepackage{bibentry}
\nobibliography*

\title{A 2019 survey of continual/incremental/lifelong learning algorithms, with a focus on suitability under application constraints}

\begin{document}
\maketitle
\section{EWC and EWC-like}
\textbf{Require knowledge of future tasks for tuning the regularization parameter $\lambda$. Plus, it is not clear how model selection is performed in many of these works, but probably in hindsight which again requires knowledge of future sub-tasks}. 
\begin{itemize}
\item \bibentry{kirkpatrick2017overcoming}
\item \bibentry{liu2018rotate}
\item \bibentry{aljundi2018memory}
\item \bibentry{chaudhry2018riemannian}
\item \bibentry{gepperth2019matrix}
\end{itemize}

\section{Incremental Moment Matching}
\textbf{Sensitive to balancing parameter $\alpha$ and L2-transfer parameter $\lambda$. Requires retention of past tasks to find a good balancing parameter $\alpha$ and knowledge of future tasks to find a good value for $\lambda$.}
\begin{itemize}
\item \bibentry{lee2017overcoming}
\end{itemize}
%
\section{Modular networks}
\textbf{Intrinsically require to know sub-task label at test time. All approaches except ExpertGate \cite{aljundi2017expertGate} ignore this. ExpertGate addresses problem by training additional sub-task classifiers but we found this to be problematic, see \cite{gepperthincremental}.}
\begin{itemize}
\item \bibentry{aljundi2017expertGate}
\item \bibentry{yoon2017lifelong}
\item \bibentry{mallya2018packnet}
\item \bibentry{rusu2016progressive}
\item \bibentry{fernando2017pathnet}
\item \bibentry{gepperthincremental}
\item \bibentry{serra2018overcoming}
\item \bibentry{li2018learning}
\end{itemize}


\section{Regularization}
\textbf{No conceptual issues with this model (although heavy use of permuted MNIST , and no statements about model selection are made).}
\begin{itemize}
\item \bibentry{aljundi2018selfless}
\end{itemize}

\section{Replay}
\textbf{No conceptual issues with these models except that amount of samples that must be generated at task $n$ is at least proportional to $n$, which renders life-long learning difficult.}
\begin{itemize}
\item \bibentry{shin2017continual}
\item \bibentry{kemker2017fearnet}
\end{itemize}

\section{Distillation}
\textbf{Distillation is a powerful tool but makes assumptions on data, i.e., new sub-tasks contain samples of old data.
This is ok in \cite{kim2018keep} since this models assumes that sub-task are just chunks of the same singe-task learning problem.  Furthermore, \cite{li2018learning} requires to know the sub-task label at test time to select the correct set of output weights (not sure about \cite{kim2018keep})}.
\begin{itemize}
\item \bibentry{shmelkov2017incremental}
\item \bibentry{li2018learning}
\item \bibentry{kim2018keep}
\end{itemize}

\section{Dual-memory}
\textbf{No issues with these, although the architectures are very complex and contain many tunable parameters of unclear impact.}
\begin{itemize}
\item \bibentry{rebuffi2017icarl}
\item \bibentry{gepperth2016bio}
\item \bibentry{kemker2017fearnet}
\end{itemize}

\nocite{*}

\clearpage
\bibliographystyle{abbrv}
\bibliography{lll}

\end{document}